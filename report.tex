\documentclass[a4paper]{article}

\usepackage{graphicx}


\begin{document}

\title{Programming Languages Assignment 2}

\author{Ellis Rourke}

\maketitle

\begin{abstract}
   Lorem ipsum dolor sit amet, consectetur adipiscing elit, sed do eiusmod
   tempor incididunt ut labore et dolore magna aliqua. Ut enim ad minim
   veniam, quis nostrud exercitation ullamco laboris nisi ut aliquip ex ea
   commodo consequat. Duis aute irure dolor in reprehenderit in voluptate
   velit esse cillum dolore eu fugiat nulla pariatur. Excepteur sint occaecat
   cupidatat non proident, sunt in culpa qui officia deserunt mollit anim id
   est laborum.
\end{abstract}

\section{Question 1}

This template report shows how to include Haskell and EBNF code into a document
and how to include the generated syntax diagrams. 



\subsection{Intro stuff 1}

Lorem ipsum dolor sit amet, consectetur adipiscing elit, sed do eiusmod
tempor incididunt ut labore et dolore magna aliqua. Ut enim ad minim
veniam, quis nostrud exercitation ullamco laboris nisi ut aliquip ex ea
commodo consequat. Duis aute irure dolor in reprehenderit in voluptate
velit esse cillum dolore eu fugiat nulla pariatur. Excepteur sint occaecat
cupidatat non proident, sunt in culpa qui officia deserunt mollit anim id
est laborum.

\subsubsection{Intro stuff 2}

Lorem ipsum dolor sit amet, consectetur adipiscing elit, sed do eiusmod
tempor incididunt ut labore et dolore magna aliqua. Ut enim ad minim
veniam, quis nostrud exercitation ullamco laboris nisi ut aliquip ex ea
commodo consequat. Duis aute irure dolor in reprehenderit in voluptate
velit esse cillum dolore eu fugiat nulla pariatur. Excepteur sint occaecat
cupidatat non proident, sunt in culpa qui officia deserunt mollit anim id
est laborum.

\section{Syntax}

Some included EBNF and syntax diagrams:

\EBNFInput{../question1/object.ebnf}

{\centering

   \includegraphics[scale=0.9]{syntax/identifier}

}

\EBNFInput{syntax/printStatement.ebnf}

{\centering

   \includegraphics[scale=0.9]{syntax/printStatement}

}

\section{Code}

Input code (plain style).



\noindent Input code (lterate style).


\end{document}
